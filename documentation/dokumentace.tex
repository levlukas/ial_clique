\documentclass[a4paper]{article}

\usepackage[inline]{enumitem}
\usepackage{gensymb}
\usepackage{caption}
\usepackage{subcaption}
\usepackage[utf8]{inputenc}
\usepackage{gensymb}
\usepackage{hyperref}
\usepackage[left=3cm,right=3cm,bottom=3.5cm]{geometry}
\usepackage{graphicx,lipsum}
\usepackage{cleveref}
\usepackage{siunitx}
\usepackage[table,xcdraw]{xcolor}
\usepackage{tabularx}
\usepackage{adjustbox}
\usepackage{array}
\usepackage{fancyhdr} % For custom headers and footers
\usepackage{graphicx}

\title{\textbf{Semestrální práce}\\Největší klika v neorientovaném grafu\\[5pt]IAL}
\author{Lukáš Lev, 256660}
%\date{19. 11. 2024}, TODO

\pagestyle{fancy}
\fancyhf{} % Clear default header and footer
\fancyhead[R]{\small\textcolor{gray}\leftmark} % Left header with the section name
\fancyfoot[R]{\thepage} % Centered footer with page number
\renewcommand{\headrulewidth}{0pt}

\begin{document}

\maketitle
\newpage

\section{Zadání} \label{sec:zadani}
Náhradní projekt je určen pouze pro studenty, kteří v předmětu IFJ neřeší souběžný projekt (např. studenti FEKT nebo studenti opakující předmět). Tento projekt je týmový a řeší jej trojice nebo čtveřice studentů.\\

\noindent
\textbf{Zadání varianty}
\textit{Klika grafu} je podgraf, který je úplným grafem (=kterýkoliv vrchol kliky je tedy spojen hranou se všemi ostatními vrcholy kliky).\\

\noindent
Vytvořte program pro hledání největší kliky v neorientovaném grafu. Pokud existuje více řešení, nalezněte všechna. Výsledky prezentujte vhodným způsobem. Součástí projektu bude načítání grafů ze souboru a vhodné testovací grafy. V dokumentaci uveďte teoretickou složitost úlohy a porovnejte ji s experimentálními výsledky.\\

\noindent
\textbf{Všeobecné informace a pokyny k náhradním projektům}

\noindent
Řešení bude vypracováno v jazyce C a bude přeložitelné (pomocí příkazu make) na serveru eva.fit.vutbr.cz. Všechny zdrojové kódy, hlavičkové soubory, testovací data aj. budou logicky separovány a uloženy v příhodně pojmenovaných podadresářích. Použití nestandardních knihoven není dovoleno. Všechny části zadání varianty jsou nutnou součástí řešení.\\

\noindent
Celkové hodnocení projektu sestává z následujících kategorií:
\begin{itemize}
    \item funkčnost implementace (až 6 bodů),
    \item projektová dokumentace (až 4 body),
    \item obhajoba (až 5 bodů).
\end{itemize}

\noindent
Řešení zabalené v jediném ZIP archivu je odevzdáváno pouze vedoucím týmu prostřednictvím STUDISu. Závazné pokyny pro vypracování projektové dokumentace a doporučení pro závěrečné obhajoby naleznete v Moodlu v sekci Projekty.
    
\section{Abstrakt} \label{sec:abstrakt}
Tento projekt byl proveden podle zadání z kapitoly \ref{sec:zadani}, jež bylo poskytnuto vyučujícím.\\

\noindent
Předmětem tohoto projektu je hledání \textbf{největší kliky v neorientovaném grafu}, což je jeden z typických problému v teorii grafů \textbf{TODO: citace}. Tato problematika je krátce popsána v kapitole \ref{sec:zadani}.\\

\noindent
Pro nalezení největší kliky v neorientovaném grafu bylo vypracováno několik algoritmů, a to sice:
\begin{itemize}
    \item Brute force algoritmus,
    \item Dynamický brute force algoritmus,
    \item Backtracking algoritmus,
    \item Branch and bound algoritmus.
\end{itemize}
Pro každý z těchto algoritmů byla stanovena časová komplexita nejprve teoreticky a následně také experimentálně pomocí jednoduchého programu v jazyce C.

\section{Teoretický úvod} \label{sec:teorie}

    
\end{document}
